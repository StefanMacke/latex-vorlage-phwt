% Abkürzungen, ggf. mit korrektem Leerraum
\newcommand{\bs}{$\backslash$\xspace}
\newcommand{\bspw}{bspw.\xspace}
\newcommand{\bzw}{bzw.\xspace}
\newcommand{\ca}{ca.\xspace}
\newcommand{\dahe}{\mbox{d.\,h.}\xspace}
\newcommand{\etc}{etc.\xspace}
\newcommand{\eur}[1]{\mbox{#1\,\texteuro}\xspace}
\newcommand{\evtl}{evtl.\xspace}
\newcommand{\ggf}{ggf.\xspace}
\newcommand{\Ggf}{Ggf.\xspace}
\newcommand{\gqq}[1]{\glqq{}#1\grqq{}}
\newcommand{\inkl}{inkl.\xspace}
\newcommand{\insb}{insb.\xspace}
\newcommand{\ua}{\mbox{u.\,a.}\xspace}
\newcommand{\usw}{usw.\xspace}
\newcommand{\Vgl}{Vgl.~}
\newcommand{\vgl}{vgl.~}
\newcommand{\zB}{\mbox{z.\,B.}\xspace}

% Befehle für häufig anfallende Aufgaben
\newcommand{\Abbildung}[1]{\autoref{fig:#1}}
\newcommand{\Anhang}[1]{\appendixname{}~\ref{#1}: \nameref{#1} \vpageref{#1}} % Referenzierung eines Anhangs
\newcommand{\includegraphicsKeepAspectRatio}[2]{\includegraphics[width=#2\textwidth,height=#2\textheight,keepaspectratio]{#1}} % Einfügen einer Graphik und Seitenverhältnis beibehalten
\newcommand{\includegraphicsRotateAndKeepAspectRatio}[2]{\includegraphics[width=#2\textheight,height=#2\textwidth,keepaspectratio,angle=90,origin=c]{#1}} % Einfügen einer entgegen dem Uhrzeigersinn gedrehten Graphik und Seitenverhältnis beibehalten
\newcommand{\itemd}[2]{\item{\bf{#1}}\\{#2}} % erzeugt ein Listenelement mit fetter Überschrift
\newcommand{\Kapitel}[1]{Kapitel~\ref{sec:#1}~(\nameref{sec:#1})} % Referenzierung eines Kapitels
\newcommand{\Abschnitt}[1]{Abschnitt~\ref{subsec:#1}~(\nameref{subsec:#1})} % Referenzierung eines Abschnitts
\newcommand{\Unterabschnitt}[1]{Unterabschnitt~\ref{subsubsec:#1}~(\nameref{subsubsec:#1})} % Referenzierung eines Unterabschnitts