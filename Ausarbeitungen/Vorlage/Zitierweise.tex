% zum Wechseln vom Chicago in den Harvard-Style müssen die Zeilen des Chicago-Styles deaktiviert und
%  die des Harvard-Styles aktiviert werden. Die gilt nur für diese Ausarbeitung.

% (z.B. \Zitati[S.~69]{Martin2008a})

% Zitate
\newcommand{\Zitati}[2][\empty]{\footnote{\Vgl \allgemeinesZitat[#1]{#2}.}} % indirektes Zitat im Chicago-Style
\newcommand{\Zitatd}[2][\empty]{\footnote{\allgemeinesZitat[#1]{#2}.}} % direktes Zitat im Chicago-Style

%\newcommand{\Zitati}[2][\empty]{ (\vgl \allgemeinesZitat[#1]{#2})} % indirektes Zitat im Harvard-Style
%\newcommand{\Zitatd}[2][\empty]{ (\allgemeinesZitat[#1]{#2})} % direktes Zitat im Harvard-Style
% nur für intern, sollte nicht in Ausarbeitungen genutzt werden:
\newcommand{\allgemeinesZitat}[2][\empty]{\ifthenelse{\equal{#1}{\empty}}{\citep{#2}}{\citep[#1]{#2}}}